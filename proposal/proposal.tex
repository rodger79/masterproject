%
%Independent Study Template.
% School of Computing
% University of Colorado at Colorado Springs
%
\documentclass[10pt]{proposal}
%\usepackage{times}
\usepackage[small,compact]{titlesec}
\usepackage[small,it]{caption}
\usepackage{url}
\usepackage{hyperref} 
\hypersetup{breaklinks} 
\usepackage{graphicx}
\usepackage{xcolor}
\usepackage{amssymb}
\usepackage{longtable}
\usepackage{pdflscape}
\usepackage{array}
\usepackage{blindtext}
\usepackage{titlesec}


% This is nice for source code listings
% \usepackage{listings}

% This is needed to include figures
\usepackage{graphicx}
\usepackage{sidecap}

% Use any additional packages you might need
\usepackage{outlines}

%
% Give values to the variables used in this document
%
\title{Push Notification Testing}
\department{Department of Computer Science\\ at the University of Colorado at Colorado Springs\\ School of Engineering and Applied Science}
\documenttype{Masters Project Proposal} %Revise for yours
\major{Software Engineering }
\proposalday{29}%Revise for yours
\proposalmonth{February}%Revise for yours
\proposalyear{2020}%Revise for yours
\author{Rodger William Byrd}
\committeechair{Kristen Walcott Justice, Advisor} %Revise for yours
\committeememberfour{Committee Member 1}
\committeememberfive{Committee Member 2}



%
% PDF Setup -- You should not need to change this
%
\hypersetup{
    colorlinks,
    linkcolor={black},
    citecolor={black},
    filecolor={black},
    urlcolor={black},
    pdftitle={\thetitle},
    pdfauthor={\theauthor},
    pdfsubject={\thedocumenttype},
    pdfkeywords={University of Colorado at Colorado Springs, \theauthor, \thedocumenttype},
    pdfstartpage={1}
}


%
% User-specified command definitions/redefinitions
%
%\newcommand{\cplusplus}{{\rm C\raise.5ex\hbox{\small ++}}}
%\setcounter{secnumdepth}{5}
%\setcounter{tocdepth}{5}

%\makeatletter
%\renewcommand\paragraph{%
%   \@startsection{paragraph}{4}{0mm}%
%      {-\baselineskip}%
%      {.5\baselineskip}%
%      {\normalfont\normalsize\bfseries}}
%\makeatother
\usepackage{wrapfig}

\newcommand{\squishlist}{
 \begin{list}{$\bullet$}
  { \setlength{\itemsep}{0pt}
     \setlength{\parsep}{3pt}
     \setlength{\topsep}{3pt}
     \setlength{\partopsep}{0pt}
     \setlength{\leftmargin}{1.5em}
     \setlength{\labelwidth}{1em}
     \setlength{\labelsep}{0.5em} } }

\newcommand{\squishlisttwo}{
 \begin{list}{$\bullet$}
  { \setlength{\itemsep}{0pt}
     \setlength{\parsep}{0pt}
    \setlength{\topsep}{0pt}
    \setlength{\partopsep}{0pt}
    \setlength{\leftmargin}{2em}
    \setlength{\labelwidth}{1.5em}
    \setlength{\labelsep}{0.5em} } }

\newcommand{\squishend}{
  \end{list}  }


\begin{document}
	
%   ==========================================================================
%   Begin front matter (pages are numbered with roman numerals)
%   ==========================================================================
  \begin{frontmatter}
       \maketitle
        %\tableofcontents
       \newpage

        % Generate the abstract	

       % \section{Abstract}
               \end{frontmatter}
       
              
\newpage    
\renewcommand*\contentsname{Contents}           
\tableofcontents{}
\newpage    

              



%   ==========================================================================
%   Begin main matter (pages are numbered with arabic numerals)
%   ==========================================================================
    \doublespacing     % Text should be double spaced
    \pagestyle{fancy}  % Turn the nice header on for the rest of the document

    %
    % I use a file for every section.  Each of these corresponds to a file
    % with the specified name ending in '.tex' (e.g., introduction.tex).
    %
  \section{Introduction}
There are major security implications to wearable devices such as smartwatches and  medical devices such as implantable pacemakers, implantable defibrillators and insulin pumps.\cite{do_is_2017}\cite{mills_wearing_2016}\cite{park_this_2016}\cite{halperin_pacemakers_2008}\cite{halperin_pacemakers_2008-1}\cite{al-sharrah_watch_2018}
In addition to the security implications of these devices, these also have the potential to cause physical harm, in the case of the medical devices.
As a first step in researching the security of medical devices, this project will focus on wearable devices with the idea that future research may be conducted on medical devices.
Our previous research has identified the push-notification process\cite{sultana_wearable_nodate} as a potential point of instability in the communication between the Android OS and wearable devices. 
This project will focus on Andriod OS and attempt to build an automated testing tool to simulate the communication and notification process between the OS and wearable devices.
The hypothesis tested will include the following impacts on the communication between the OS and wearable devices:
\begin{enumerate}
 \item Available device storage is low
 \item Missing patches and updates 
 \item Device carriers result in performance deltas
\end{enumerate}
Previous research was manual and potential error could have been introduced. 
This project will attempt to provide more precise findings in support of this research by creating a simulation and testing platform that will allow varying hypothesis to be tested.


  \section{Background and Related Work}
\subsection{Background}
Previous research by Sultana\cite{sultana_wearable_nodate} showed that some Android devices had delayed notifications on paired wearable devices. 
Her research was conducted by performing a small manual study measuring the time it took from calling a phone to the time the notification showed up on the wearable device.
That research found that there were significant delays in some of the testing scenarios and they varied by device and operating system.
Some of the potential causes noted in that research were missing patches and updates and limited available storage on devices.
The results from the findings showed that the performance variations were due to Andriod OS and not the wearable devices themselves.

\subsection{Related Work}
Wearable and medical devices are more of a security concern then other types of devices. 
They are generally more personal then other devices such as a pc or smartphone. 
They contain health and medical information and a lot of the same sensitive information that is on a smartphone. 
Most users think of them as connected specifically to their phone, but the devices have many other connections such as wifi, bluetooth, usb, and SMS capabilities to exfiltrate data.
Do et al. showed that they could get root access to samsung gear devices using a custom bootloader and were able to access sensitive information, such as SMS information, contact information and biomedical data.\cite{do_is_2017}
In a related study, Al-Sharrah et al. showed that Apple watches store contact details, text messages, calendar details, Emails, pictures, and wallet data including stored payment cards.\cite{al-sharrah_watch_2018}
 % \input{Impact}
 % \input{Discussion} 
  \section{Proposed Work}
The proposed work is to test push notificaitons in Android devices. 
\subsection{Initial Phase}
\paragraph{Emulation}
This will involve using Android Studio to emulate and test notifications to determine what can cause delays.
Use debugging tools to determine what can cause delays in notification in andriod devices and wearable devices.
\paragraph{Simulation}
Simulates criteria that may lead to delays, such has low storage, high memory usage, high processor usage to attempt to determine factors that can lead to delays.
Review Android OS patching history to determine what fixes have been put in place related to notifications and wearable devices.
\subsection{Second Phase}
Test findings from initial phase on actual hardware. %6
 % \input{evaluation}
  \section{Completed Work}
Previous research by Sultana\cite{sultana_wearable_nodate} showed that some Android devices had delayed notifications on paired wearable devices. 
Some of the potential causes noted in that research   %3.5
 % \input{relatedwork} %1.5
  \section{Tasks and Timeline}
\begin{outline}
	\1 Emulation of Android OS
		\2 Install Android Studio
		\2 Setup Andriod Studio environemnt
		\2 Choose operating systems to test
		\2 Build app to emulate push notifications?
		\2 Research for existing software for testing android apps
	\1 Emulation of Wear OS
		\2 Test capability of Android Studio to simulate interaction between mobile and wear environments
	\1 Create automated test to test multiple scenarios quickly
		\2 Integrate test environment with Andriod Studio Emulator
	\1 Possible downloadable app for real world testing
		\2 TBD
	\1 Test bluetooth interference on notification delays
\end{outline} %0.5
 % \section{Conclusion}
 % \input{appendix} 



%   ==========================================================================
%   Wrap up the document with the Bibliography (looks for the specified .bib)
%   ==========================================================================
\pagebreak    

\makebibliography{sources}
%\bibliographystyle{references}
%bibliography{references}

%\input{appendix} %If you need appendices

\end{document}

\section{Background and Related Work}
\subsection{Background}
Previous research by Sultana\cite{sultana_wearable_nodate} showed that some Android devices had delayed notifications on paired wearable devices. 
Her research was conducted by performing a small manual study measuring the time it took from calling a phone to the time the notification showed up on the wearable device.
That research found that there were significant delays in some of the testing scenarios and they varied by device and operating system.
Some of the potential causes noted in that research were missing patches and updates and limited available storage on devices.
The results from the findings showed that the performance variations were due to Andriod OS and not the wearable devices themselves.

\subsection{Related Work}
Wearable and medical devices are more of a security concern then other types of devices. 
They are generally more personal then other devices such as a pc or smartphone. 
They contain health and medical information and a lot of the same sensitive information that is on a smartphone. 
Most users think of them as connected specifically to their phone, but the devices have many other connections such as wifi, bluetooth, usb, and SMS capabilities to exfiltrate data.
Do et al. showed that they could get root access to samsung gear devices using a custom bootloader and were able to access sensitive information, such as SMS information, contact information and biomedical data.\cite{do_is_2017}
In a related study, Al-Sharrah et al. showed that Apple watches store contact details, text messages, calendar details, Emails, pictures, and wallet data including stored payment cards.\cite{al-sharrah_watch_2018}
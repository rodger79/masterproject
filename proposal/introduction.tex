\section{Introduction}
There are major security implications to wearable devices such as smartwatches and  medical devices such as implantable pacemakers, implantable defibrillators and insulin pumps.\cite{do_is_2017}\cite{mills_wearing_2016}\cite{park_this_2016}\cite{halperin_pacemakers_2008}\cite{halperin_pacemakers_2008-1}\cite{al-sharrah_watch_2018}
In addition to the security implications of these devices, these also have the potential to cause physical harm, in the case of the medical devices.
As a first step in researching the security of medical devices, this project will focus on wearable devices with the idea that future research may be conducted on medical devices.
Our previous research has identified the push-notification process\cite{sultana_wearable_nodate} as a potential point of instability in the communication between the Android OS and wearable devices. 
This project will focus on Andriod OS and attempt to build an automated testing tool to simulate the communication and notification process between the OS and wearable devices.
The hypothesis tested will include the following impacts on the communication between the OS and wearable devices:
\begin{enumerate}
 \item Available device storage is low
 \item Missing patches and updates 
 \item Device carriers result in performance deltas
\end{enumerate}
Previous research was manual and potential error could have been introduced. 
This project will attempt to provide more precise findings in support of this research by creating a simulation and testing platform that will allow varying hypothesis to be tested.

